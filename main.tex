\documentclass[9pt]{ctexbeamer}

\usepackage{pgfpages}

\makeatletter 
\def\beamer@framenotesbegin{% at beginning of slide
    \usebeamercolor[fg]{normal text}
}
\makeatother
%%%%%%%%%%%%%%%%%%

%%%%-----导入宏包-----%%%%
\usepackage{style/shubeamer}
\usepackage{xeCJK}
\usepackage{amsmath,amsfonts,amssymb,bm}
\usepackage{color}
\usepackage{graphicx,hyperref,url}
\usepackage{comment}
\usepackage{xcolor}
\usepackage{subcaption}
\usepackage{booktabs}
\usepackage{graphbox}
\usepackage{arydshln}
\usepackage[super,square]{natbib}
\usepackage{wrapfig} % 文字绕排
\usepackage{calligra}
%%%%%%%%%%%%%%%%%%

%%%%-----设置字体-----%%%%
\setmainfont{CMU Serif}
\setsansfont{CMU Sans Serif}
\usefonttheme{professionalfonts}
%%%%%%%%%%%%%%%%%%

\definecolor{nvidia}{RGB}{102,156,28}
\definecolor{red}{RGB}{184,13,73}
\definecolor{darkred}{RGB}{145,12,7}
\definecolor{orange}{RGB}{242,151,36}
\definecolor{darkteal}{RGB}{43,106,108}
\definecolor{darkgrey}{RGB}{64,64,64}
\definecolor{darkblue}{RGB}{4,37,58}
\definecolor{tan}{RGB}{225,221,191}
\definecolor{green}{RGB}{76,131,122}
\definecolor{darkgreen}{RGB}{42,50,46}
\definecolor{bluegray}{RGB}{33,36,39}
\definecolor{brown}{RGB}{110,54,42}

% 设置 Beamer 主题
\beamertemplateballitem
\AtBeginSection[]
{
  \begin{frame}<beamer>
    \frametitle{\textbf{目录}}
    \textbf{\tableofcontents[currentsection]}
  \end{frame}
}
%%%%%%%%%%%%%%%%%%

%%%%----设定图片存放路径
\graphicspath{{figures/}}

%%%%----首页信息设置----%%%% 方括号内容将显示在边栏
\AtBeginDocument{%
%%%%----标题
\title[报告标题]{\fontsize{13pt}{18pt}\selectfont {报告标题}}
%%%%----副标题
\subtitle{\fontsize{9pt}{14pt}\selectfont \textbf{Title of Report}}
%%%%----个人信息设置
\author[杜镕远]{%
  \begin{tabular}{ll}%
    答辩人: & 杜镕远 \tabularnewline%
    学\quad 号:   & 19000000 \tabularnewline%
    专\quad 业:   & 摸鱼与应用摸鱼 \tabularnewline%
  \end{tabular}
}
%%%%----机构信息
\institute[SHU]{上海大学}
%%%%----日期信息
\date[\today]{\today}}
%%%%----Macros for Vector, Matrix, Tensor, Math Operator and Misc
\input{style/macros.tex}
%%%%%%%%%%%%%%%%%%

%%%%----开始正文
\begin{document}


%% 封面信息,方括号内容是显示在左侧边栏的内容
\begin{frame}
  \titlepage
\end{frame}

%% 提纲页
\section*{目录}
\label{sec:toc}

\begin{frame}
  \frametitle{\textbf{目录}}
  \textbf{\tableofcontents}
\end{frame}
%% section* 目录 (end)

\section{研究内容及进度}

\subsection{课题主要研究内容}

\begin{frame}{课题主要研究内容}
  \begin{figure}
    \includegraphics[width=0.4\linewidth]{SHU-blue}
    \caption{课题主要研究内容}
  \end{figure}
\end{frame}

\subsection{进度介绍}

\begin{frame}{进度介绍}
  \begin{figure}
    \includegraphics[width=0.4\linewidth]{SHU-blue}
    \caption{进度介绍}
  \end{figure}
\end{frame}

\section{研究工作及成果}

\begin{frame}{已完成的研究工作及成果}
  \begin{block}{已完成的研究工作简介}
    \begin{itemize}
      \setlength{\itemsep}{6pt}
      \item XXXX
      \item XXXX
      \item XXXX
      \item XXXX
      \item XXXX
      \item XXXX
    \end{itemize}
  \end{block}
\end{frame}

\subsection{研究工作一}

\begin{frame}{研究工作一}
  \begin{block}{无编号公式}
    无编号公式示例:
    $$
      k:[-\pi,\pi] \rightarrow [0,1]
    $$
  \end{block}
\end{frame}

\subsection{研究工作二}

\begin{frame}{研究工作二}
  \begin{block}{有编号公式}
    \begin{itemize}
      \item 有编号公式示例:输入为图像 
      \begin{equation}
        \vx\in\reals^{C_{\text{in}}\times H\times W}
      \end{equation}
      其中 $C_{\text{in}}$ 表示通道, $H$ 表示图像高度, $W$ 表示图像深度.
    \end{itemize}
  \end{block}
\end{frame}

\subsection{研究工作三}

\begin{frame}{研究工作三}
  \begin{block}{表格}
    表格示例, 如表 \ref{tab:unique_values} 所示. 

    \begin{table}[htbp]
      \small
      \centering
      \caption{train.csv 每列非重复元素个数}
      \label{tab:unique_values}
      \begin{tabular}{lc}
        \toprule
        column & \# unique values \\
        \midrule
        posting\_id & 34250 \\
        image & 32412 \\
        image\_phash & 28735 \\
        title & 33117 \\
        label\_group & 11014 \\
        \bottomrule
      \end{tabular}
    \end{table}
  \end{block}
\end{frame}

\subsection{研究工作四}

\begin{frame}{研究工作四}
  \begin{block}{并排图片}
    并排图片示例.
  \end{block}
  \begin{figure}[htbp]
    \centering
    \begin{minipage}[t]{0.48\textwidth}
      \centering
      \includegraphics[width=2.5cm]{SHU-blue.pdf}
      \caption{并排图片1}
      \label{fig:left_side}
    \end{minipage}
    \begin{minipage}[t]{0.48\textwidth}
      \centering
      \includegraphics[width=2.5cm]{SHU-blue.pdf}
      \caption{并排图片2}
      \label{fig:right_side}
    \end{minipage}
  \end{figure}
\end{frame}

\section{后期工作与安排}

\subsection{后期研究工作}

\begin{frame}{后期研究工作}
  \begin{block}{后期研究工作}
    \begin{itemize}
      \setlength{\itemsep}{6pt}
      \item XXXX
      \item XXXX
      \item XXXX
      \item XXXX
    \end{itemize}
  \end{block}
\end{frame}

\subsection{进度安排}

\begin{frame}{进度安排}
  \begin{block}{进度安排}
    \begin{itemize}
      \setlength{\itemsep}{6pt}
      \item XXXX\cite{bottou2018optimization}
      \item XXXX\cite{POLYAK19641}
      \item XXXX\cite{nesterov1983method}
      \item XXXX\cite{Goodfellow-et-al-2016}
    \end{itemize}
  \end{block}
\end{frame}

\section{问题与解决方案}

\begin{frame}{问题与解决方案}
  \begin{block}{问题}
    \begin{itemize}
      \item XXXX
    \end{itemize}
  \end{block}
  \begin{block}{对应解决方案}
    \begin{itemize}
      \item XXXX
    \end{itemize}
  \end{block}
\end{frame}

\section{按时完成可能性}

\begin{frame}{按时完成可能性}
  \begin{block}{按时完成可能性}
    \begin{itemize}
      \setlength{\itemsep}{6pt}
      \item XXXX 
      \item XXXX 
      \item XXXX 
      \item XXXX 
    \end{itemize}
  \end{block}
\end{frame}



\appendix

\begin{frame}[allowframebreaks]{参考文献}
  \bibliographystyle{style/gbt7714-2005}
  \bibliography{reference.bib} 
\end{frame}

\begin{frame}
	\begin{center}
    {\Huge\calligra Thanks for your attention!}
    \vspace{1cm}

    {\Huge Q \& A}
  \end{center}
\end{frame}

\end{document}
